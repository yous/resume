\documentclass[a4paper,10pt]{article}

%A Few Useful Packages
\usepackage{marvosym}
\usepackage{fontspec} 					%for loading fonts
\usepackage{xunicode,xltxtra,url,parskip} 	%other packages for formatting
\RequirePackage{color,graphicx}
\usepackage[usenames,dvipsnames]{xcolor}
\usepackage[big]{layaureo} 				%better formatting of the A4 page
% an alternative to Layaureo can be ** \usepackage{fullpage} **
\usepackage{supertabular} 				%for Grades
\usepackage{titlesec}					%custom \section
% Korean
\usepackage{kotex}

%Setup hyperref package, and colours for links
\usepackage{hyperref}
\definecolor{linkcolour}{rgb}{0,0.2,0.6}
\hypersetup{colorlinks,breaklinks,urlcolor=linkcolour, linkcolor=linkcolour}

%FONTS
\defaultfontfeatures{Mapping=tex-text}
%\setmainfont[SmallCapsFont = Fontin SmallCaps]{Fontin}
%%% modified for Karol Kozioł for ShareLaTeX use
\setmainfont[
  SmallCapsFont = Fontin-SmallCaps.otf,
  BoldFont = Fontin-Bold.otf,
  ItalicFont = Fontin-Italic.otf
]
{Fontin.otf}
%%%

%CV Sections inspired by:
%http://stefano.italians.nl/archives/26
\titleformat{\section}{\Large\scshape\raggedright}{}{0em}{}[\titlerule]
\titlespacing{\section}{0pt}{3pt}{3pt}
%Tweak a bit the top margin
%\addtolength{\voffset}{-1.3cm}

%Italian hyphenation for the word: ''corporations''
\hyphenation{im-pre-se}

%-------------WATERMARK TEST [**not part of a CV**]---------------
\usepackage[absolute]{textpos}

\setlength{\TPHorizModule}{30mm}
\setlength{\TPVertModule}{\TPHorizModule}
\textblockorigin{2mm}{0.65\paperheight}
\setlength{\parindent}{0pt}

%--------------------BEGIN DOCUMENT----------------------
\begin{document}

%WATERMARK TEST [**not part of a CV**]---------------
%\font\wm=''Baskerville:color=787878'' at 8pt
%\font\wmweb=''Baskerville:color=FF1493'' at 8pt
%{\wm
%	\begin{textblock}{1}(0,0)
%		\rotatebox{-90}{\parbox{500mm}{
%			Typeset by Alessandro Plasmati with \XeTeX\  \today\ for
%			{\wmweb \href{http://www.aleplasmati.comuv.com}{aleplasmati.comuv.com}}
%		}
%	}
%	\end{textblock}
%}

\pagestyle{empty} % non-numbered pages

\font\fb=''[cmr10]'' %for use with \LaTeX command

%--------------------TITLE-------------
\par{\centering
  {\Huge 유차영
}\bigskip\par}

%--------------------SECTIONS-----------------------------------
%Section: Personal Data
\section{인적사항}

\begin{tabular}{rl}
  \textsc{웹사이트:} & \url{https://yous.be} \\
  \textsc{GitHub:}   & \href{https://github.com/yous}{@yous} \\
\end{tabular}

%Section: Education
\section{학력}
\begin{tabular}{rl}
  2011.02--2017.02 & KAIST \\
                   & \textsc{수리과학과}, \textsc{전산학부} 학사과정 \\
  2009.03--2011.02 & 대전과학고등학교
\end{tabular}

%Section: Work Experience at the top
\section{경력}
\begin{tabular}{r|p{11cm}}
  \textsc{2016.01--현재} & \textsc{Twitter} \\
                           & 한국어 번역 모더레이터 \\
  \multicolumn{2}{c}{} \\

  \textsc{2015.11--현재} & \textsc{Lionbridge} \\
                           & 영한 번역 프리랜서 \\
  \multicolumn{2}{c}{} \\

  \textsc{2015.06--2016.03} & \textsc{GoldSpoon} \\
                            & Meteor와 React를 사용한 \href{http://groov.fm}{Groov} 웹 개발 \\
  \multicolumn{2}{c}{} \\

  \textsc{2013.07--2014.12} & \textsc{탐생}, 서울특별시 \\
                            & 안드로이드 앱 \href{https://play.google.com/store/apps/details?id=kr.co.tamseng.StudyHelper}{스터디 헬퍼} 개발 총괄 \\
  \multicolumn{2}{c}{} \\

  \textsc{2011--2013} & \textsc{KAIST CERT} \\
                      & 학생 팀 \\
  \multicolumn{2}{c}{} \\

  \textsc{2011--현재} & \textsc{GoN} \\
                        & KAIST 정보보안 및 해킹 동아리 \\
\end{tabular}

\section{활동}
\begin{tabular}{r|p{11cm}}
  \textsc{2014.12.25} & \textsc{ChristmasCTF 2014} \\
                      & 2위 수상 \\
  \multicolumn{2}{c}{} \\

  \textsc{2012.07.26--29} & \textsc{해킹 콘퍼런스 DEFCON 20 본선}, Las Vegas \\
                          & \emph{KAIST GoN} \\
  \multicolumn{2}{c}{} \\

  \textsc{2012} & \textsc{제11회 KAIST-POSTECH 학생대제전}, 포항시 \\
  \multicolumn{2}{c}{} \\

  \textsc{2011} & \textsc{제10회 KAIST-POSTECH 학생대제전}, 대전광역시 \\
\end{tabular}

%Section: Scholarships and additional info
%\section{수상경력 및 장학금}
%\begin{tabular}{rl}
%  2011 봄--2012 봄 & 국가우수장학금(이공계)
%\end{tabular}

%Section: Computer Skills
%\section{기술}
%\begin{description}
  \item[Programming Languages] \hfill
    \begin{itemize}
      \item \href{https://www.ruby-lang.org/}{Ruby}
      \item Android
      \item JavaScript
    \end{itemize}
  \item[Web Frameworks] \hfill
    \begin{itemize}
      \item \href{http://rubyonrails.org}{Ruby on Rails}
    \end{itemize}
  \item[Tools] \hfill
    \begin{itemize}
      \item \href{http://www.vim.org}{Vim}
      \item \href{http://git-scm.com}{Git}
      \item \href{https://github.com/ruby/rake}{Rake}
      \item {\fb\LaTeX}\setmainfont[SmallCapsFont=Fontin-SmallCaps.otf]{Fontin.otf}
    \end{itemize}
\end{description}


%Section: Open Source Projects
\section{오픈 소스 프로젝트}
\subsection{Owned}
\begin{description}
  \item[whiteglass] \url{https://github.com/yous/whiteglass} \\
    간결한 반응형 Jekyll 테마
  \item[vim-open-color] \url{https://github.com/yous/vim-open-color} \\
    \href{https://yeun.github.io/open-color/}{Open Color}를 이용한 어두운 Vim 컬러스킴
  \item[Arcus Docker] \url{https://github.com/yous/arcus-docker} \\
    \href{https://github.com/naver/arcus}{Arcus Cache Cloud}를 위한 Docker
  \item[TweetDeck Image Extension] \url{https://github.com/yous/tweetdeck_image_extension} \\
    \href{https://tweetdeck.twitter.com}{TweetDeck}의 추가 이미지 미리보기 지원
  \item[homebrew-fadedrubies] \url{https://github.com/yous/homebrew-fadedrubies} \\
    이전 버전 루비를 위한 Homebrew 탭
  \item[vanilli.sh] \url{https://github.com/yous/vanilli.sh} \\
    가볍고 보편적으로 사용 가능한 셸 설정
  \item[Pinpoint Docker] \url{https://github.com/yous/pinpoint-docker} \\
    \href{https://github.com/naver/pinpoint}{Pinpoint}를 위한 Docker
  \item[BaseHangul] \url{https://github.com/yous/basehangul} \\
    한글 바이너리 인코더 라이브러리
  \item[YousList] \url{https://github.com/yous/YousList} \\
    Adblock Plus, uBlock Origin, 1Blocker, AdAway를 위한 차단 필터
  \item[aheui.vim] \url{https://github.com/yous/aheui.vim} \\
    Vim 아희 문법 강조 플러그인
  \item[lime] \url{https://github.com/yous/lime} \\
    간단한 Zsh 테마
  \item[Raheui] \url{https://github.com/yous/raheui} \\
    루비로 작성한 \href{http://aheui.github.io}{아희} 인터프리터
  \item[Themespace] \url{https://github.com/themespace} \\
    \href{http://octopress.org}{Octopress} 테마 데모 사이트
\end{description}

\subsection{Contributing}
\begin{description}
  \item[Pinpoint] \url{https://github.com/naver/pinpoint} \\
    Gson 플러그인 추가
  \item[nodejs.org] \url{https://github.com/nodejs/nodejs.org} (Contributor) \\
    \href{https://github.com/orgs/nodejs/teams/website}{@nodejs/website} 팀 소속
  \item[nodeschool] \url{https://github.com/nodeschool} (Contributor) \\
    \href{https://github.com/orgs/nodeschool/teams/seoul}{@nodeschool/seoul} 팀 소속
  \item[jews] \url{https://github.com/disjukr/jews} (Contributor) \\
    여러 뉴스 사이트를 위한 파서 작성
  \item[Loco] \url{https://github.com/khwon/loco} \\
    리팩토링, 테스트와 문서화 작업
  \item[RuboCop] \url{https://github.com/bbatsov/rubocop} \\
    다양한 Cop의 자동 수정 기능 추가
  \item[whitespace] \url{https://github.com/lucalsew/whitespace} \\
    레이아웃 개선 및 다양한 소셜 프로필 추가
  \item[AndroidSlidingUpPanel] \url{https://github.com/umano/AndroidSlidingUpPanel} \\
    위에서 아래로 슬라이드 하는 기능 추가
\end{description}

\subsection{Translation}
\begin{description}
  \item[ruby/www.ruby-lang.org] \url{https://github.com/ruby/www.ruby-lang.org} (Contributor) \\
    \href{https://github.com/orgs/ruby/teams/www-ruby-lang-org-i18n-ko}{@ruby/www.ruby-lang.org-i18n-ko} 팀 소속
  \item[ruby-korea] \url{https://github.com/ruby-korea} (Contributor) \\
    번역 리뷰
  \item[nodejs-ko] \url{https://github.com/nodejs/nodejs-ko} (Contributor) \\
    글 번역, 번역 리뷰
  \item[dalzony/ruby-style-guide] \url{https://github.com/dalzony/ruby-style-guide} (Contributor) \\
    번역 리뷰
\end{description}

%\newpage
%\hypertarget{gmat}{\textsc{Gmat}\setmainfont{LMRoman10 Regular}\textregistered\setmainfont[SmallCapsFont=Fontin-SmallCaps]{Fontin-Regular}}

%\XeTeXpdffile ''GMAT.pdf'' page 1 scaled 800

\end{document}
