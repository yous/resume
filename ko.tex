\documentclass[a4paper,10pt]{article}

%A Few Useful Packages
\usepackage{marvosym}
\usepackage{fontspec} 					%for loading fonts
\usepackage{xunicode,xltxtra,url,parskip} 	%other packages for formatting
\RequirePackage{color,graphicx}
\usepackage[usenames,dvipsnames]{xcolor}
\usepackage[big]{layaureo} 				%better formatting of the A4 page
% an alternative to Layaureo can be ** \usepackage{fullpage} **
\usepackage{supertabular} 				%for Grades
\usepackage{titlesec}					%custom \section
% Korean
\usepackage{kotex}

%Setup hyperref package, and colours for links
\usepackage{hyperref}
\definecolor{linkcolour}{rgb}{0,0.2,0.6}
\hypersetup{colorlinks,breaklinks,urlcolor=linkcolour, linkcolor=linkcolour}

%FONTS
\defaultfontfeatures{Mapping=tex-text}
%\setmainfont[SmallCapsFont = Fontin SmallCaps]{Fontin}
%%% modified for Karol Kozioł for ShareLaTeX use
\setmainfont[
SmallCapsFont = Fontin-SmallCaps.otf,
BoldFont = Fontin-Bold.otf,
ItalicFont = Fontin-Italic.otf
]
{Fontin.otf}
%%%

%CV Sections inspired by:
%http://stefano.italians.nl/archives/26
\titleformat{\section}{\Large\scshape\raggedright}{}{0em}{}[\titlerule]
\titlespacing{\section}{0pt}{3pt}{3pt}
%Tweak a bit the top margin
%\addtolength{\voffset}{-1.3cm}

%Italian hyphenation for the word: ''corporations''
\hyphenation{im-pre-se}

%-------------WATERMARK TEST [**not part of a CV**]---------------
\usepackage[absolute]{textpos}

\setlength{\TPHorizModule}{30mm}
\setlength{\TPVertModule}{\TPHorizModule}
\textblockorigin{2mm}{0.65\paperheight}
\setlength{\parindent}{0pt}

%--------------------BEGIN DOCUMENT----------------------
\begin{document}

%WATERMARK TEST [**not part of a CV**]---------------
%\font\wm=''Baskerville:color=787878'' at 8pt
%\font\wmweb=''Baskerville:color=FF1493'' at 8pt
%{\wm
%	\begin{textblock}{1}(0,0)
%		\rotatebox{-90}{\parbox{500mm}{
%			Typeset by Alessandro Plasmati with \XeTeX\  \today\ for
%			{\wmweb \href{http://www.aleplasmati.comuv.com}{aleplasmati.comuv.com}}
%		}
%	}
%	\end{textblock}
%}

\pagestyle{empty} % non-numbered pages

\font\fb=''[cmr10]'' %for use with \LaTeX command

%--------------------TITLE-------------
\par{\centering
{\Huge 유차영
}\bigskip\par}

%--------------------SECTIONS-----------------------------------
%Section: Personal Data
\section{인적사항}

\begin{tabular}{rl}
  \textsc{웹사이트:}  & \url{http://yous.be} \\
  \textsc{GitHub:}    & \href{https://github.com/yous}{@yous} \\
  \textsc{Bitbucket:} & \href{https://bitbucket.org/yous}{@yous}
\end{tabular}

%Section: Work Experience at the top
\section{경력}
\begin{tabular}{r|p{11cm}}
  \textsc{2013.07-2014.12.31} & \textsc{탐생}, 서울특별시 \\
  & \emph{CTO} \\
  & 안드로이드 앱 \href{https://play.google.com/store/apps/details?id=kr.co.tamseng.StudyHelper}{스터디 헬퍼} 개발 총괄. \\
  \multicolumn{2}{c}{} \\

  \textsc{2012.07.26--29} & \textsc{해킹 콘퍼런스 DEFCON 20 본선}, Las Vegas \\
  & \emph{KAIST 정보보안 및 해킹동아리 GoN 팀원} \\
  & CTF 참가.
\end{tabular}

%Section: Education
\section{학력}
\begin{tabular}{rl}
  2011.02--현재 & KAIST, 대전광역시 \\
  & \textsc{수리과학과} (복수전공: \textsc{전산학과}) 학사과정 \\
  2009.03--2011.02 & 대전과학고등학교, 대전광역시
\end{tabular}

%Section: Scholarships and additional info
%\section{수상경력 및 장학금}
%\begin{tabular}{rl}
%  2011 봄--2012 봄 & 국가우수장학금(이공계)
%\end{tabular}

%Section: Computer Skills
\section{기술}
\begin{description}
  \item[Programming Languages] \hfill
    \begin{itemize}
      \item \href{https://www.ruby-lang.org/}{Ruby}
      \item Android
      \item JavaScript
    \end{itemize}
  \item[Web Frameworks] \hfill
    \begin{itemize}
      \item \href{http://rubyonrails.org}{Ruby on Rails}
    \end{itemize}
  \item[Tools] \hfill
    \begin{itemize}
      \item \href{http://www.vim.org}{Vim}
      \item \href{http://git-scm.com}{Git}
      \item \href{https://github.com/ruby/rake}{Rake}
      \item {\fb\LaTeX}\setmainfont[SmallCapsFont=Fontin-SmallCaps.otf]{Fontin.otf}
    \end{itemize}
\end{description}


%Section: Open Source Projects
\section{오픈 소스 프로젝트}
\begin{description}
  \item[Pinpoint] \url{https://github.com/naver/pinpoint} (Volunteer)
    \begin{itemize}
      \item Pinpoint is an open source APM (Application Performance Management) tool for large-scale distributed systems written in Java.
      \item Gson 플러그인 추가.
    \end{itemize}
  \item[ruby/www.ruby-lang.org] \url{https://github.com/ruby/www.ruby-lang.org} (Contributor)
    \begin{itemize}
      \item Source of the \url{https://www.ruby-lang.org} website.
      \item \href{https://github.com/orgs/ruby/teams/www-ruby-lang-org-i18n-ko}{@ruby/www.ruby-lang.org-i18n-ko} 팀 소속.
    \end{itemize}
  \item[ruby-korea] \url{https://github.com/ruby-korea} (Contributor)
    \begin{itemize}
      \item 루비 프로그래밍 언어 한글 문서
      \item \href{https://github.com/ruby-korea/rubygems-guides}{rubygems-guide} 번역 리뷰.
    \end{itemize}
  \item[iojs-ko] \url{https://github.com/nodejs/iojs-ko} (Contributor)
    \begin{itemize}
      \item io.js 한국 커뮤니티 \url{http://nodejs.github.io/iojs-ko/}
      \item 번역 리뷰.
    \end{itemize}
  \item[new.nodejs.org] \url{https://github.com/nodejs/new.nodejs.org} (Contributor)
    \begin{itemize}
      \item The Node.js website.
      \item \href{https://github.com/orgs/nodejs/teams/website}{@nodejs/website} 팀 소속.
    \end{itemize}
  \item[nodeschool] \url{https://github.com/nodeschool} (Contributor)
    \begin{itemize}
      \item Open source community and tools for education powered by node.js.
      \item \href{https://github.com/orgs/nodeschool/teams/seoul}{@nodeschool/seoul} 팀 소속.
    \end{itemize}
  \item[BaseHangul] \url{https://github.com/yous/basehangul} (Owner)
    \begin{itemize}
      \item Human-readable binary encoding, \href{https://basehangul.github.io}{BaseHangul} for Ruby.
      \item 한글 바이너리 인코더 라이브러리.
    \end{itemize}
  \item[jews] \url{https://github.com/disjukr/jews} (Contributor)
    \begin{itemize}
      \item just news.
      \item 뉴스 본문 외의 정보를 전부 제거한 뒤 페이지를 재구성하는 스크립트.
    \end{itemize}
  \item[aheui.vim] \url{https://github.com/yous/aheui.vim} (Owner)
    \begin{itemize}
      \item Vim 아희 문법 강조 플러그인.
    \end{itemize}
  \item[Raheui] \url{https://github.com/yous/raheui} (Owner)
    \begin{itemize}
      \item 루비로 작성한 \href{http://aheui.github.io}{아희} 인터프리터.
    \end{itemize}
  \item[Loco] \url{https://github.com/khwon/loco} (Volunteer)
    \begin{itemize}
      \item KAIST 로코 BBS.
      \item 리팩토링, 테스트와 문서화 작업.
    \end{itemize}
  \item[RuboCop] \url{https://github.com/bbatsov/rubocop} (Volunteer)
    \begin{itemize}
      \item 커뮤니티 루비 스타일 가이드에 기반한 루비 정적 코드 분석기.
    \end{itemize}
  \item[Themespace] \url{https://github.com/themespace} (Owner)
    \begin{itemize}
      \item \href{http://octopress.org}{Octopress} 테마 데모 사이트.
    \end{itemize}
  \item[whitespace] \url{https://github.com/lucalsew/whitespace} (Volunteer)
    \begin{itemize}
      \item \href{http://octopress.org}{Octopress}를 위한 미니멀 화이트 테마.
    \end{itemize}
  \item[AndroidSlidingUpPanel] \url{https://github.com/umano/AndroidSlidingUpPanel} (Volunteer)
    \begin{itemize}
      \item 위로 슬라이드 할 수 있는 패널 라이브러리.
      \item 위에서 아래로 슬라이드 하는 기능 추가.
    \end{itemize}
\end{description}

%\newpage
%\hypertarget{gmat}{\textsc{Gmat}\setmainfont{LMRoman10 Regular}\textregistered\setmainfont[SmallCapsFont=Fontin-SmallCaps]{Fontin-Regular}}

%\XeTeXpdffile ''GMAT.pdf'' page 1 scaled 800

\end{document}
